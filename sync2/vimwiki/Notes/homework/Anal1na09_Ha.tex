
\documentclass{report}

\usepackage{enumitem}
\usepackage[utf8]{inputenc} \usepackage[T1]{fontenc} \usepackage{textcomp}
\usepackage{amsmath, amssymb}
\usepackage{mathtools}

\newcommand\myeq{\stackrel{\mathclap{\normalfont\mbox{{IV}}}}{=}}
\newcommand\swapi{\stackrel{\mathclap{\normalfont\mbox{{I$\leftrightarrow$}II}}}{$\implies$}}

% \author{\huge{Van Hoa Nguyen\\MN:0483979\\Tutor: Rohde-Gouromichos Victor Panos  }}
\title{\Huge{Analysis 1 und Lineare Algebra}}
\author{\huge{Van Hoa Nguyen}\\MN: 0483979}
\date{\today}

\begin{document}

\maketitle
% \newpage% or \cleardoublepage
% \pdfbookmark[<level>]{<title>}{<dest>}
% \pagebreak


\chapter*{09. Hausaufgabenblatt}

\section*{1. Aufgabe}
\textbf{Zeigen Sie mit vollstandgier Induktion:}
\subsection*{1.1 Aufgabe}
Für alle natürlichen Zahlen $n \geq 1$ gilt.

$$\sum_{k=1}^{n}k^{3} = \frac{n^{2}\cdot(n+1)^{2}}{4}$$

\begin{enumerate}

    \item \textbf{IA:} $n_0=1$ 
        \[
            \sum_{k=1}^{1}k^{3} = \frac{1^{2}\cdot(1+1)^{2}}{4}
            \Rightarrow 1=\frac{1\cdot4}{1}
            \Rightarrow 1=1
        \]
        $\rightarrow A(n_0)$ gilt
    \item \textbf{IV:} Für ein beliebige, festes n gilt:
        $$\sum_{k=1}^{n}k^{3} = \frac{n^{2}\cdot(n+1)^2}{4}$$

    \item \textbf{IB:}  Dann gilt für n+1:
        $$\sum_{k=1}^{n+1}k^{3} = \frac{(n+1)^{2}\cdot(((n+1)+1)^2}{4}$$
        $$\Rightarrow\sum_{k=1}^{n+1}k^{3} = \frac{(n+1)^{2}\cdot(n+2)^2}{4}$$

    \item \textbf{IS:}  
        \[
            \sum_{k=1}^{n+1}k^{3} \myeq \sum_{k=1}^{n}k^{3} + (n+1)^{3}
        \]
        $$\Rightarrow \sum_{k=1}^{n}k^{3}= \frac{n^{2} \cdot (n+1)^2}{4} + \frac{4(n+1)^{3}}{4}$$

        $$\Rightarrow \sum_{k=1}^{n}k^{3}= \frac{n^{2} \cdot (n+1)^2+{4(n+1)^{3}}}{4}$$

        $$\Rightarrow \sum_{k=1}^{n}k^{3}= \frac{(n+1)^{2}\cdot(n^2+4n+4))}{4}$$

        $$\Rightarrow \sum_{k=1}^{n}k^{3}= \frac{(n+1)^{2}\cdot(n+2)^2}{4}$$

        $\rightarrow$ Die Gleichung gilt für alle natürlichen Zahlen $n \geq 1$

\end{enumerate}

\subsection*{1.2 Aufgabe}
Für all $n \in \mathbb N, n \geq 1,$ gilt dass $n^{2}+n$ gerade ist.

\begin{enumerate}
    \item \textbf{IA:} $n_0=1$ (funktioniert auch $n_0=0$, da 0 eine gerade Zahl ist)
        $$1^{2}+1 = 2 $$ $\rightarrow$ ist gerade, $A(n_0)$ gilt 
    \item \textbf{IV:} Für ein beliebiges, festes n gilt: $n^{2}+n$ ist 
        gerade bzw. $$n^{2}+n = 2m; n, m \in \mathbb N$$
    \item \textbf{IB:} Dann gilt für n+1:
        \[
            (n+1)^{2}+n+1 = 2m; m, n \in \mathbb N
        \]
    \item \textbf{IS:}
        $$(n+1)^{2}+n+1 = n^{2}+n+2n+2$$
        $$\Rightarrow \underline{n^{2}+n}+2n+2 \myeq \underline{2m}+2n+2$$
        $$\Rightarrow 2m+2n+2=2m+2(n+1)$$
        $$\Rightarrow 2m+2n+2=2[m+(n+1)]$$
        $(m+n+1)$ substituieren wir mit $k \in \mathbb N$,
        dann ergibt sich $2k$. Es gilt: eine natürliche Zahl
        multipliziert mit 2 ist immer gerade. \newline 
        $\rightarrow $ $n^{2}+n, $ ist gerade für alle $n\in \mathbb N, n\geq 1$

\end{enumerate}

\section*{2. Aufgabe}
\begin{enumerate}
    \item Der Nullvektor gehört zu T, $\vec{0} \in T$, 
        denn die Null-Funktion erfüllt für alle x die Bedingung f(1)=0
    \item Seien $f,g \in T$, dann muss auch gelten: $f+g \in T$
        \newline $f(1)=0$ und $g(1)=0$ $\Rightarrow$ $f(1)+g(1)=0+0=0$ \newline
        $(f+g)(1)= f(1)+g(1)= 0$ \newline
        Somit gilt: $f+g \in T$ 
    \item Seien $f \in T$ und $\lambda \in \mathbb R$, dann muss auch gelten: $\lambda f \in T$ \newline
        $\lambda \cdot f(1) = \lambda \cdot 0 = 0$ \newline
        $\lambda f$ erfüllt die Bedingung f(1) = 0 $\Rightarrow \lambda k \in T $ 
\end{enumerate}
T erfüllt all 3 Eigenschaften und ist somit ein Unterraum von V
\section*{3. Aufgabe}
$B_1=\{\vec{e_1},\vec{e_2},\vec{e_3}\}$ und $B_2=\{\vec{v_1}\vec{v_2}\vec{v_3}\}$

$$
\vec{v_1}= \begin{bmatrix}
    1 \\
    0 \\
    0 \\
    \end{bmatrix}, \vec{v_2}= \begin{bmatrix}
    2 \\
    1 \\
    -1 \\
    \end{bmatrix}, \vec{v_3}= \begin{bmatrix}
    -2 \\
    1 \\
    4 \\
\end{bmatrix}, 
$$ 
\subsection*{3.1 Aufgabe}
\begin{enumerate}
    \item Spalte von $f_{B_1,B_2}$
        $$f_{B_1,B_2}(e_1) = K_{B_1}(f(K_{B_1}^{-1}(e_1))) $$
        $$\Rightarrow f_{B_1,B_2}(e_1) = K_{B_1}(f(\begin{bmatrix}
            1 \\
            0 \\ 
            0
        \end{bmatrix})) $$
        
        $$\Rightarrow f_{B_1,B_2}(e_1) = K_{B_1}(\begin{bmatrix}
            1 \\
            0 \\ 
            -1
        \end{bmatrix}) $$

        $$\Rightarrow f_{B_1,B_2}(e_1) = \begin{bmatrix}
            1 \\
            0 \\ 
            -1
        \end{bmatrix} $$
        weil 
        \[
           \begin{bmatrix}
           1 \\
           0 \\
           -1
           \end{bmatrix} = 1 
           \begin{bmatrix}
              1 \\
              0 \\
              0
           \end{bmatrix}   + 0 
           \begin{bmatrix}
              0 \\
              1 \\
              0
           \end{bmatrix}   -1  
           \begin{bmatrix}
              0 \\
              0 \\
              1
           \end{bmatrix}    
        \] 

    \item Spalte von $f_{B_1,B_2}$
        $$f_{B_1,B_2}(e_2) = K_{B_1}(f(K_{B_1}^{-1}(e_2))) $$
        $$\Rightarrow f_{B_1,B_2}(e_2) = K_{B_1}(f(\begin{bmatrix}
            0 \\
            1 \\ 
            0
        \end{bmatrix})) $$

        $$\Rightarrow f_{B_1,B_2}(e_2) = K_{B_1}(\begin{bmatrix}
            -2 \\
            -1 \\ 
            1
        \end{bmatrix}) $$

        $$\Rightarrow f_{B_1,B_2}(e_2) = \begin{bmatrix}
        -2 \\
            -1 \\ 
            1
        \end{bmatrix} $$

        weil 
        \[
           \begin{bmatrix}
            -2 \\
            -1 \\
            1 
           \end{bmatrix} = -2 
           \begin{bmatrix}
              1 \\
              0 \\
              0
           \end{bmatrix}   -1
           \begin{bmatrix}
              0 \\
              1 \\
              0
           \end{bmatrix}   1  
           \begin{bmatrix}
              0 \\
              0 \\
              1
           \end{bmatrix}    
        \] 

    \item Spalte von $f_{B_1,B_2}$
        $$f_{B_1,B_2}(e_3) = K_{B_1}(f(K_{B_1}^{-1}(e_3))) $$

        $$\Rightarrow f_{B_1,B_2}(e_3) = K_{B_1}(f(\begin{bmatrix}
            0 \\
            0 \\ 
            1
        \end{bmatrix})) $$

        $$\rightarrow f_{b_1,b_2}(e_3) = k_{B_1}(\begin{bmatrix}
           -2 \\
            1 \\ 
            4 
        \end{bmatrix}) $$

        $$\rightarrow f_{b_1,b_2}(e_3) = \begin{bmatrix}
           -2 \\
            1 \\ 
            4 
        \end{bmatrix} $$

        weil 
        \[
           \begin{bmatrix}
           -2 \\
            1 \\
            4 
           \end{bmatrix} = -2 
           \begin{bmatrix}
              1 \\
              0 \\
              0
           \end{bmatrix}   +1 
           \begin{bmatrix}
              0 \\
              1 \\
              0
           \end{bmatrix}   +4 
           \begin{bmatrix}
              0 \\
              0 \\
              1
           \end{bmatrix}    
        \] 

\end{enumerate}
\[ \rightarrow f_{B_1, B_1} =
   \begin{bmatrix}
       1 & -2 & -2 \\
       0 & -1 & 1 \\
       -1 & 1 & 4
   \end{bmatrix} 
\]
\subsection*{3.2 Aufgabe}
\[
    T_{B_2,B_1} = \left\lbrack K_{B_1}(\begin{bmatrix}
        1 \\
        0 \\
        -1
    \end{bmatrix}) K_{B_1}(\begin{bmatrix}
        2 \\ 
        1 \\
        -1
    \end{bmatrix}) K_{B_1}(\begin{bmatrix}
        -2 \\
        1 \\
        4
    \end{bmatrix})\right\rbrack = \begin{bmatrix}
    1 & 2 & -2 \\
    0 & 1 & 1 \\
    -1 & -1 & 4
    \end{bmatrix}
\] 
Weil wie wir gesehen haben, dass wenn die Basis des Koordinatenvektors
die Standardbasis ist, gilt $K_{B_1}(\vec{v})=\vec{v}$

\subsection*{3.3 Aufgabe}

Für $T_{B_1,B_2}$ = $(T_{B_2,B_1})^{-1}$

\[
\begin{bmatrix}
    1 & 2 & -2  &\bigm| & 1 & 0 & 0 \\
    0 & 1 & 1   &\bigm| & 0 & 1 & 0 \\
    -1 & -1 & 4  &\bigm| & 0 & 0 & 1 \\ 
\end{bmatrix} \implies
\begin{bmatrix}
    1 & 0 & -4  &\bigm| & 1 & -2 & 0 \\
    0 & 1 & 1   &\bigm| & 0 & 1 & 0 \\
    -1 & -1 & 4  &\bigm| & 0 & 0 & 1 \\ 
\end{bmatrix} 
\]
\[
\implies\begin{bmatrix}
    1 & 0 & -4  &\bigm| & 1 & -2 & 0 \\
    0 & 1 & 1   &\bigm| & 0 & 1 & 0 \\
    0 & -1 & 0  &\bigm| & 1 & -2 & 1 \\ 
\end{bmatrix} \implies
\begin{bmatrix}
    1 & 0 & -4  &\bigm| & 1 & -2 & 0 \\
    0 & -1 & 0  &\bigm| & 1 & -2 & 1 \\ 
    0 & 1 & 1   &\bigm| & 0 & 1 & 0 \\
\end{bmatrix}
\]
\[
\implies\begin{bmatrix}
    1 & 0 & -4  &\bigm| & 1 & -2 & 0 \\
    0 & 1 & 0   &\bigm| & -1 & 2 & -1 \\
    0 & 0 & 1  &\bigm| & 1 & -1 & 1 \\ 
\end{bmatrix} \implies
\begin{bmatrix}
    1 & 0 & 0  &\bigm| & 5 & -6 & 4 \\
    0 & 1 & 0  &\bigm| & -1 & 2 & -1 \\ 
    0 & 0 & 1   &\bigm| & 1 & -1 & 1 \\
\end{bmatrix}
\]
\[
T_{B_1,B_2} = \begin{bmatrix}
    5 & -6 & 4 \\
    -1 & 2 & -1 \\
    1 & -1 & 1 \\ 
\end{bmatrix}
\]

Unterschritte:
\begin{enumerate}
    \item I-2II
    \item III-I
    \item II$\leftrightarrow$III
    \item II+III und -1 * II
    \item I + 4III
\end{enumerate}

\subsection*{3.4 Aufgabe}

$f_{B_1,B_2} = id_{B_1,B_2} \cdot f_{B_1,B_1} \cdot id_{B_2,B_1}$
\[
    \begin{bmatrix}
        5 & -6 & 4 \\
        -1 & 2 & -1 \\
        1 & -1 & 1
    \end{bmatrix} \cdot
    \begin{bmatrix}
        1 & -2 & -2 \\
        0 & -1 & 1 \\
        -1 & 1 & 4
    \end{bmatrix} \cdot 
    \begin{bmatrix}
        1 & 2 & -2 \\
        0 & 1 & 1 \\
        -1 & -1 & 4
    \end{bmatrix} =
    \begin{bmatrix}
        1 & 2 & -2 \\
        0 & -1 & -1 \\
        -1 & -1 & 4 \\
    \end{bmatrix}
\]

\section*{4. Aufgabe}



\end{document}

